%%%%%%%%%%%%%%%%%%%%%%%%%%%%%%%%%%%%%%%%%%%%%%%%%%%%%%%%%%%%%%%%%%%%%%%%%%%%%%%%
% The Legrand Orange Book
% LaTeX Template
% Version 2.2 (30/3/17)
%
% This template has been downloaded from:
% http://www.LaTeXTemplates.com
%
% Original author:
% Mathias Legrand (legrand.mathias@gmail.com) with modifications by:
% Vel (vel@latextemplates.com)
%
% License:
% CC BY-NC-SA 3.0 (http://creativecommons.org/licenses/by-nc-sa/3.0/)
%
% Compiling this template:
% This template uses biber for its bibliography and makeindex for its index.
% When you first open the template, compile it from the command line with the
% commands below to make sure your LaTeX distribution is configured correctly:
%
% 1) pdflatex main
% 2) makeindex main.idx -s StyleInd.ist
% 3) biber main
% 4) pdflatex main x 2
%
% After this, when you wish to update the bibliography/index use the appropriate
% command above and make sure to compile with pdflatex several times
% afterwards to propagate your changes to the document.
%
% This template also uses a number of packages which may need to be
% updated to the newest versions for the template to compile. It is strongly
% recommended you update your LaTeX distribution if you have any
% compilation errors.
%
% Important note:
% Chapter heading images should have a 2:1 width:height ratio,
% e.g. 920px width and 460px height.
%
%%%%%%%%%%%%%%%%%%%%%%%%%%%%%%%%%%%%%%%%%%%%%%%%%%%%%%%%%%%%%%%%%%%%%%%%%%%%%%%%

%----------------------------------------------------------------------------------------
%	DOCUMENT CONFIGURATIONS
%----------------------------------------------------------------------------------------

\documentclass[12pt,a4paper]{book} % Default font size and left-justified equations

\input{structure} % Insert the stucture.tex file which contains the majority of the structure behind the template

\begin{document}

%----------------------------------------------------------------------------------------

%----------------------------------------------------------------------------------------
%	TITLE PAGE
%----------------------------------------------------------------------------------------

\begingroup
\thispagestyle{empty}
\begin{tikzpicture}[remember picture,overlay]
\node[inner sep=0pt] (background) at (current page.center) {\includegraphics[width=\paperwidth]{./Pictures/background.png}};
\draw (current page.center) node [fill=ocre!30!white,fill opacity=0.6,text opacity=1,inner sep=1cm]{\Huge\centering\bfseries\sffamily\parbox[c][][t]{\paperwidth}{\centering Java\\[15pt] % Book title
{\Large Compte Rendu de Travaux Pratiques}\\[20pt] % Subtitle
{\huge Axel LE BOT }}}; % Author name
\end{tikzpicture}
\vfill
\endgroup
%----------------------------------------------------------------------------------------
%	COPYRIGHT PAGE
%----------------------------------------------------------------------------------------

\newpage
~\vfill
\thispagestyle{empty}

\noindent Copyright \copyright\ 2017-2018 Axel LE BOT \\ % Copyright notice

\noindent Licensed under the Creative Commons Attribution-NonCommercial 3.0 Unported License (the ``License''). You may not use this file except in compliance with the License. You may obtain a copy of the License at \url{http://creativecommons.org/licenses/by-nc/3.0}. Unless required by applicable law or agreed to in writing, software distributed under the License is distributed on an \textsc{``as is'' basis, without warranties or conditions of any kind}, either express or implied. See the License for the specific language governing permissions and limitations under the License.\\ % License information

%----------------------------------------------------------------------------------------
%	INTRODUCTION
%----------------------------------------------------------------------------------------

\pagestyle{empty} % No headers

\chapterimage{./Pictures/cover-start}

\chapter*{Introduction}\addcontentsline{toc}{part}{\texorpdfstring{\protect\@myparttocformat{Introduction}}{Introduction}}

Ce rapport retrace mon travail sur les différentes scéances de travaux pratiques dans le cadre de l'apprentissage du langage Java. Ces Travaux Pratiques nous ont permit d'utiliser plus concretements la programmation orientée objet.

Le Java est un langage orientée objet créé par Oracle et dérivant du C/C++. Il a été implémenter dans le but que le programme compiler une fois peut être lancé depuis n'importe quel envrionnement compatible Java sans recompilation, ce principe est décrit par WORE ("Write Once Run Anywhere").\\
Les application Java sont compiler en \textit{bytecode} pouvant être lancé depuis n'importe quelle JVM (Java Virtual Machine) indépendemment de l'architecture système.

Pour information ce langage est utilisé dans le SDK Android qui a valut à Google une poursuite en justice.

Nous définissons ici les différentes notions de Java :
\begin{itemize}
  \item JDK (Java Development Kit) : Framework Java implémentant des classes de basses. Actuellement en version 9.
  \item JRE (Java Runtime Environment) : Sert à lancer des programme Java. (contient : JVM + Java Packages Classes + Runtime Library)
  \item JVM (Java Virtual Machine) : La machine virtuel dans lequel le programme sera lancé et gère le cycle de vie des programmes.
\end{itemize}

% Java est aussi une ile d'Indonésie d'environ 139,000 km carré, mais nous sortons ici du cadre de travail.

\cleardoublepage % Forces the table of contents chapter to start on an odd page so it's on the right

\pagestyle{fancy} % Print headers again

%----------------------------------------------------------------------------------------
%	TABLE OF CONTENTS
%----------------------------------------------------------------------------------------

%\usechapterimagefalse % If you don't want to include a chapter image, use this to toggle images off - it can be enabled later with \usechapterimagetrue

\chapterimage{./Pictures/cover-table_of_contents} % Table of contents heading image

\pagestyle{empty} % No headers

\tableofcontents % Print the table of contents itself

\cleardoublepage % Forces the first chapter to start on an odd page so it's on the right

\pagestyle{fancy} % Print headers again

%----------------------------------------------------------------------------------------
%	TABLE OF FIGURES
%----------------------------------------------------------------------------------------

\chapterimage{./Pictures/cover-table_of_contents} % Table of contents heading image

\pagestyle{empty} % No headers

\listoffigures % Print the table of contents itself

\cleardoublepage % Forces the first chapter to start on an odd page so it's on the right

\pagestyle{fancy} % Print headers again

%----------------------------------------------------------------------------------------
%	PART ONE
%----------------------------------------------------------------------------------------

\part{TP}

\chapterimage{./Pictures/cover-gear}
\chapter{TP1 : Méthodes de classe et méthodes d'instance}
\textit{L’objectif de ce premier TP est de nous familiariser avec la notion de classe, notion essentielle de la programmation en Java, en nous faisant créer plusieurs classes illustrant une situation concrète et notamment d’insister sur la différence entre méthodes de classe et méthode d’instance.}

\section{Exercice 1 : Classe Ville}
\textit{L'objectif de cet exercice est de créer une classe Ville possédant un nom, une superficie et une population. Un constructeur par défaut ains qu'un constructeur prenant en paramètre les trois attributs cité précedemment ainsi que les accesseurs et mutateurs de chaques attributs seront ajoutés à la classe. Nous allons également surcharger la méthode toString() afin d'afficher la description d'une Ville.}
\inputminted[linenos,firstline=5,lastline=90]{java}{../sources/src/tp1/Ville.java}

\section{Exercice 2 : Classe Departement}
\textit{L'objectif de cet exercice est de créer une clase Département, permettant de regrouper plusieurs villes. Cette classes possèdera comme attributs, un nom, un numéro, un nombre de villes saisies et donc un tableau de Villes. Un constructeur sera ajouté à la classe ainsi qu'une méthode ajouterVille, permettant d'ajouter une Ville au tableau de Villes du Département. Nous allons également surcharger la méthode toString() afin d'afficher la description d'un département et ses villes qu'il contient.}
\inputminted[linenos,firstline=3,lastline=80]{java}{../sources/src/tp1/Departement.java}

\section{Exercice 3 : Classe Main}
\textit{L'objectif de cet exercice est de créer un classe Main contenant une méthode main dans laquelle nous allons instancier les différentes classes nécessaire au scénario proposé.}
\inputminted[linenos,firstline=3,lastline=29]{java}{../sources/src/tp1/Main.java}

\begin{figure}[H]
  \centering
  \includegraphics[width=350pt]{./tp/Pictures/tp1-execute}
  \caption{Exécution TP1}
  \label{Exécution TP1}
\end{figure}

\section{Exercice 4 : Attributs et méthodes de classe vs attributs et méthodes d'instances}
\textit{L'objectif de cet exercice est de manipulier le mot clef \mintinline{java}{static} afin différencier les membres de classe et les membres d'instance.}
\\\\
On remarque que si l'on rajoute le mot clef \mintinline{java}{static} à l'attribut nom de la classe Département, toutes les instances de Département auront le même nom.\\
En effet le mot clef \mintinline{java}{static} permet de spécifier qu'un membre appartient à tout une classe plutôt qu'une instance, c'est à dire que seulement une instance d'attribut static sera existante même si l'on crée un centaine d'instance de la classe.\\
Les deux méthodes \mintinline{java}{estIdentiqueA(Ville v)} et \mintinline{java}{sontIdentiques(Ville v1, Ville v2)} permettent de comparer deux Ville.\\
On implémente ainsi ces méthodes :

\inputminted[linenos,firstline=58,lastline=69]{java}{../sources/src/tp1/Ville.java}

\begin{itemize}
  \item \mintinline{java}{estIdentiqueA(Ville v)} de comparer une instance de Ville depuis une autre instance de Ville.
  \item \mintinline{java}{sontIdentique(Ville v1, Ville v2)} permet de comparer deux instance de Ville depuis la classe Ville.
\end{itemize}

Nous allons faciliter la comparaison de deux instance en surchargeant la méthode \mintinline{java}{equals} ainsi que la méthode \mintinline{java}{hashCode} de la manière suivante :
\inputminted[linenos,firstline=71,lastline=88]{java}{../sources/src/tp1/Ville.java}

\chapterimage{./Pictures/cover-stack}
\chapter{TP2 : Les Listes}
\textit{L'objectif de ce TP est de découvrir les listes avec l'utilisation de la classe \mintinline{java}{ArrayList}. On créera comme au TP1 des classes, ce qui nous permettra d'être encore plus à l'aise avec la programmation objet en JAVA}

\section{Exercice 1 : Classe Cours et tri de liste}
\textit{L'objectif de cet exercice est de créer une classe Cours. Cette classes possèdera comme attributs, un code, un intitulé et un volume horraire. Nous implémeterons un constructeur et nous surchargerons la méthode toString.}
\inputminted[linenos,firstline=3,lastline=61]{java}{../sources/src/tp2/Cours.java}

On ajoutera la méthode \mintinline{java}{verifyCode(String code)} afin d'enlever les caractère non alpha-numérique mis par l'utilisateur lors de l'instanciation de la classe ou encore lors de la mutation de l'attibut Code.
\inputminted[linenos,firstline=58,lastline=60]{java}{../sources/src/tp2/Cours.java}

\section{Exercice 2 : Classe Formation}
\textit{L'objectif de cet exercice est de créer une classe Formation, permettant de regrouper plusieurs cours. Cette classes possèdera comme attributs, un code, un intitulé et une liste de cours. Nous implémeterons un constructeur et nous surchargerons la méthode toString.}
\inputminted[linenos,firstline=5,lastline=66]{java}{../sources/src/tp2/Formation.java}

\section{Exercice 3 : Classe Main}
\textit{L'objectif de cet exercice est de simuler le scénario proposé en utilisant les classes créés précédemment. Pour cela, nous devrons comprendre et utiliser les méthodes fournies par la classe ArrayList.}
\\\\
Je crée la classe Main suivante :
\inputminted[linenos,firstline=9,lastline=46]{java}{../sources/src/tp2/Main.java}

On obtient l'exécution suivante :
\begin{figure}[H]
  \centering
  \includegraphics[width=300pt]{./tp/Pictures/tp2-execute}
  \caption{Exécution TP2}
  \label{Exécution TP2}
\end{figure}

\section{Exercice 4 : Trie de liste}
\textit{L'objectif de cet exercice est d'apprendre à implémenter les comportement d'une interface simple permettant de comparer les instances d'une classe et ainsi trier plus facilement les instances dans une collection par exemple.}
\\\\
Afin de pouvoir trier les objects d'un tableau nous devons surcharger la méthode \mintinline{java}{compareTo(Object o)} de l'interface \mintinline{java}{Comparable}.\\
Une interface permet à une classe d'implémenter un compartement réutilisable. Pour celà il faut utiliser le mot clef \mintinline{java}{implements} suivit de l'interface souhaiter pour implementer les méthodes de l'interface, ainsi pour l'interface \mintinline{java}{Comparable} servant à comparer les instances de la classe Cours nous feront comme ci-dessous :
\inputminted[linenos,firstline=3,lastline=3]{java}{../sources/src/tp2/Cours.java}

Ainsi nous pourront surcharger les méthodes de l'interface, ici la méthode \mintinline{java}{CompareTo(Object o)} qui comparera un maximum d'attributs de la classe comme ci-dessous :
\inputminted[linenos,firstline=52,lastline=63]{java}{../sources/src/tp2/Cours.java}

On utilisera de préférence la méthode \mintinline{java}{sort()} de l'instance plutôt que celle de la classe mère Collection, celà évitera les érreurs de type de Collection.\\
Nous pouvons si nous le voulons créer un comparateur à la voler en passant en paramètre \mintinline{java}{new Comparator(){}}, mais nous n'en avons pas l'utiliter ici.
\inputminted[linenos,firstline=43,lastline=43]{java}{../sources/src/tp2/Main.java}

\section{Synthèse personnelle}

Il existe plusieurs type de collection, facilitant toutes des manipulations précise.
\begin{itemize}
  \item ArrayList : permet à une collection d'assurer le stockage de donnés potentiellement dupliqués.
  \item HashSet : permet à une collection d'assurer l'unicité des donnés quelle contient.
  \item HashMap : permet de faire un dictionnaire.
  \item LinkedHashMap : permet de faire un dictionnaire ordonné.
  \item TreeMap : permet de faire un dictionnaire trié.
\end{itemize}

\begin{figure}[H]
  \centering
  \includegraphics[width=400pt]{./tp/Pictures/tp2-collection-cheatsheet}
  \caption{Collection Cheatsheet}
  \label{Collection Cheatsheet}
\end{figure}

Comme nous avons pu le voir les Collections ont plusieurs avantages sur les tableaux. Nous en listons ci-dessous une liste exhaustive :
\begin{itemize}
  \item Muable
  \item Une API exhaustive contrairement aux tableaux.
  \item Peut être utilisé sur plusieurs threads sans danger.
  \item Accepte ou interdire les éléments null.
  \item Peut avoir des vues (unmodifiable, subList, filter...)
  \item Les méthodes \mintinline{java}{equals}, \mintinline{java}{hashCode} et \mintinline{java}{toString} d'une collection réponse aux attentes d'un utilisateur, ces même méthodes sur un tableaux sont souvent sujet à des bugs.
\end{itemize}

\chapterimage{./Pictures/cover-inheritance}
\chapter{TP3 : Héritage et polymorphisme}
\textit{}

\chapterimage{./Pictures/cover-abstract}
\chapter{TP4 : Classes abstraites}
\textit{L'objectif de ce TP numéro 4 est de définir et de manipuler une classe abstraite. Une telle classe ne permet pas d'instancier des objets. Elle sert de classe de base pour un héritage. Les méthodes abstraites sont juste déclarées (les signatures des dites méthodes sont fournies) dans les classes abstraites. Leur définition est donnée dans les classes filles.}

\section{Exercice 1 : Cercle et rectangle}
\textit{L'objectif de cet exercice est de créer deux classes Cercle et Rectangle, ces deux classes ayant de nombreux points communs, il est préférable de créer une classe FormeGeometrique dite "abstraite" donc les deux classes hériteraient.}
\\\\
Je crée dans un premier temps la classe abstraite FormeGeometrique, voici son code :
\inputminted[linenos,firstline=3,lastline=23]{java}{../sources/src/tp4/FormeGeometrique.java}
Cette classe est une classe abstraite nous ne pouvons pas l'instancier comme nous instancierons une classe normal :
\begin{minted}{java}
FormeGeometrique forme = new FormeGeometrique(1.0);
\end{minted}
Cependant nous pouvons instancier une classe anonyme de cette facon :
\begin{minted}{java}
FormeGeometrique forme = new FormeGeometrique(1.0) {
            @Override
            public double calculPerimetre() {
                return 0;
            }

            @Override
            public double calculSuperficie() {
                return 0;
            }

            @Override
            public String toString() {
                return "$classAnonymous{" +
                        "epaisseur=" + epaisseur +
                        '}';
            }
        };
\end{minted}

Je crée ensuite les classes Cercle et Rectangle de cette manière :
\inputminted[linenos,firstline=3,lastline=33]{java}{../sources/src/tp4/Cercle.java}

\inputminted[linenos,firstline=3,lastline=38]{java}{../sources/src/tp4/Rectangle.java}

Ces deux classes ne sont pas abraites et les méthodes sont bien redéfinies, il est donc possible de les instancier.

\section{Exercice 2 : Tableau de formes géométriques}
\textit{L'objectif de cet exercice est de créer un tableau de formes géométriques, on pourra ainsi voir l'intérêts supplémentaires de l'héritage et des classes abstraites.}
\\\\
Je crée donc une classe TableauFormeGeometrique, avec comme attribut un nombre de forme ainsi qu'une liste contenant des formes géométriques.
Voici le code de la classe TableauFormeGeometrique :
\inputminted[linenos,firstline=6,lastline=36]{java}{../sources/src/tp4/TableauFormeGeometrique.java}

Voici le code de la classe Main :
\inputminted[linenos,firstline=3,lastline=38]{java}{../sources/src/tp4/Main.java}

Le fait d’avoir créé une classe abstraite FormeGeometrique nous permet de stocker dans ce notre tableau des cercles et des rectangles puisque les deux héritent de FormeGéométrique, c’est l’un des intérêts des classes abstraites.

À l'éxecution On obtient l’affichage final suivant :
\begin{figure}[H]
  \centering
  \includegraphics[width=500pt]{./tp/Pictures/tp4-execute}
  \caption{Exécution TP4}
  \label{Exécution TP4}
\end{figure}

\chapterimage{./Pictures/cover-manipulate}
\chapter{TP5 : Description et manipulation d’objets}
\textit{}


%----------------------------------------------------------------------------------------
%	CONCLUSION
%----------------------------------------------------------------------------------------

\cleardoublepage % Forces the conclusion chapter to start on an odd page so it's on the right

\pagestyle{empty}

\chapterimage{./Pictures/cover-end}

\chapter*{Conclusion}\addcontentsline{toc}{part}{\texorpdfstring{\protect\@myparttocformat{Conclusion}}{Conclusion}}

\lipsum[1-2]

\pagestyle{fancy}

%----------------------------------------------------------------------------------------

\end{document}
